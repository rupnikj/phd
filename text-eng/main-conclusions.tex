%--------------------------------------------------------------------------------------------------
%
\chapter{Conclusions}\label{chap:conclusions}
%--------------------------------------------------------------------------------------------------

\section{Discussion}

In the thesis we study a generalization of CCA to more than two
sets of variables. We present a new result that proves that
the complexity of the SUMCOR problem 
is NP-hard and describe a novel approach to finding several sets
of nonlinear patterns, based on a locally convergent method.  
Experimentally, we observe that the
performance of the local method (with linear convergence) is
generally good, although we identify problem settings where the
local method can be far from globally optimal. We present a novel
SDP relaxation of the problem, which can be used to obtain new
local solutions and to provide several new computationally tractable bounds on
global optimality of the SUMCOR problem solutions. The
usefulness of the bounds is explored on synthetic problem
instances and a problems related to cross-lingual text
mining. We introduce a new preprocessing step based on random
projections to reduce the dimensionality of high dimensional problems
such as in document corpora, making memory requirements tractable.

We present an application of two generalizations of CCA, the
SUMCOR and SSCOR formulations to cross-lingual similarity function
learning. The cross-lingual similarity functions are applied to 
the task of cross-lingual cluster linking, where we present and evaluate a novel
approach that combines features based on semantic and language analysis.
The approach is shown to be scalable both in 
terms of number of articles and number of languages, while accurately linking events.

On the task of mate retrieval, we observe that refining the LSI-based 
projections with hub CCA leads to improved retrieval precision, but the 
methods perform comparably on the task of event linking. Further inspection 
showed that the CCA-based approach reached a higher precision on smaller 
clusters. The interpretation is that the linking features are highly 
aggregated for large clusters, which compensates the lower per-document 
precision of LSI. Another possible reason is that the advantage that we 
show on Wikipedia is lost on the news domain. This hypothesis could be 
validated by testing the approach on documents from a different domain.

The experiments show that the hub CCA-based features present a good baseline, 
which can greatly benefit from additional semantic-based features. Even though 
in our experiments the addition of CCA-based features to semantic features did not 
lead to great performance improvements, there are two important benefits in the 
approach. First, the linking process can be sped up by using a smaller set of
candidate clusters. Second, the approach is robust to languages where semantic 
extraction is not available, due to scarce linguistic resources.

\section{Future Work}

\bolded{SUMCOR Problem.}
The experiments indicate that the noisy 1-dimensional embeddings present difficulties for the Horst
algorithm, which is in contrast to the performance on random generic problem instances. A natural
question is, are there other problem structures that result in suboptimal behavior of the local approach? 

Our empirical results focus on text data, and an interesting direction is to extend the analysis to 
data from other modalities, such as images, sensor streams and graphs.

Also of interest is the complexity analysis of the other generalizations proposed in \cite{Kettenring}.

\bolded{Cluster Linking.}
The proposed cross-lingual analysis approaches represent an important building block in our
approach to cross-lingual cluster linking. The language component is 
built independently from the cluster linking component. 
It is possible that better embeddings can be obtained by methods that
jointly optimize a classification task and the embedding.

Another point of interest is to evaluate our approach to cluster-linking on languages with scarce 
linguistic resources, where semantic annotation might not be available. For this purpose, 
the labelled dataset of linked clusters should be extended first. The mate retrieval evaluation 
shows that even for language pairs with no training set overlap, the hub CCA recovers some signal.

In order to further improve the performance of the classifier for cluster linking, additional features should also
be extracted from articles and clusters and checked if they can increase the accuracy of the classification.
Since the amount of linguistic resources vary significantly from language to language it would also make sense
to build a separate classifier for each language pair. Intuitively, this should improve performance since weights
of individual learning features could be adapted to the tested pair of languages.
