%--------------------------------------------------------------------------------------------------
%
\chapter{Notation and Definitions}\label{chap:notation}

We first introduce the notation we use throughout the thesis:
\begin{itemize}
\item Column vectors are denoted by lowercase letters, e.g. $x$ and matrices are denoted by uppercase letters, e.g. $X$.
\item Subscripts are used to enumerate vectors or matrices, e.g. $x_1, x_2$, $X_1$, except in the
special case of the identity matrix, $I_n$ and the zero matrix $0_{k,l}$. In these cases, the subscripts denote row and column dimensions.
\item We use superscripted symbol $T$ for vector and matrix transpose, e.g. $x^T$
\item Let $\norm{v}$ or $\norm{v}_2$ denote the $\ell_2$ norm of the vector $v$ and $\norm{A}_F$, $\norm{A}_1$ and $\norm{A}_2$ to denote the Frobenius norm and operator norms induced by $1$-norm and $2$-norm respectively.
\item MATLAB notation~\cite{golub}
\begin{itemize}
\item The $i$-th element of vector $x$ is denoted by $x(i)$ and the matrix entry in the $i$-th row and $j$-th column is denoted by $X(i,j)$.
\item The $i$-th row of matrix $X$ is denoted by $X(i,:)$ and the $j$-th column by $X(:,j)$.
 matrix elements, rows and columns {(e.g. ${X(i,j), X(i,:), X(:,j)}$)}
\item Matrix concatenation: $[A B]$ represents horizontal concatenation and $[A; B]$ represents vertical concatenation.
\item $diag(v)$ denotes a diagonal matrix whose diagonal entries correspond to vector $v$.
\item $\vec{1}_k$ denotes a column vector with $k$ elements all equal to $1$.
\end{itemize}
\item Spaces
\begin{itemize}
 \item $\RR^n$ denotes the $n$-dimensional real vector space
 \item $\RR^{n\times m}$ denotes the $(n \cdot m)$-dimensional vector space used when specifying
 matrix dimensions.
 \item $\NN$ denotes the natural numbers.
 \item $\sym_n^{+}$ denotes the space of symmetric positive definite $n$-by-$n$ matrices.
\end{itemize}
\item Random vectors are denoted by calligraphic letters, e.g. $\mathcal{X}$ and $\mathcal{X} \in \RR^n$ denotes their dimension.
\end{itemize}


\section{Sample Datasets and the Multiview Assumption}
The following definitions will be relevant for our discussion of kernel versions of the methods relevant to this thesis.

\begin{definition}\label{def:notation:sample_dataset}
A \emph{sample dataset} with $\ell$ samples and $n$ dimensions is a set
$$ S := \lbrace x_1, \ldots, x_\ell \rbrace, $$
 where $x_i \in \RR^n$ are  generated independently and identically distributed (i.i.d.) according to
an underlying distribution.
\end{definition}

\begin{definition}\label{def:notation:sample_matrix}
A \emph{$n\times\ell$ sample matrix} based on a dataset $S$ with $\ell$ samples and $n$ dimensions is obtained by horizontal concatenation:
$$ X := \left[ x_1 \cdots x_\ell \right]. $$
\end{definition}

\begin{definition}\label{def:notation:multiview_dataset}
A \emph{multiview sample dataset} with $\ell$ samples and $m$ views is a set:
$$ S = \big\{ \left( x_1^{(1)},\ldots, x_1^{(m)} \right), \ldots, \left( x_\ell^{(1)}, \ldots, x_\ell^{(m)} \right) \big\}, $$
where $x_i^{(j)} \in \RR^{n_j}$ corresponds to the $j$-th view of the $i$-th sample.
We assume that each sample point was generated independently and identically distributed (i.i.d.)
according to an underlying distribution with a specific structure.
We assume that the samples represent different views of an underlying object, that is, the observed random vectors
are functions of an unobserved random vector:
  $$ \left( \mathcal{X}^{(1)}, \ldots, \mathcal{X}^{(m)} \right) = \left( f_1(\mathcal{O}), \ldots, f_m(\mathcal{O}) \right).$$
\end{definition}

\begin{definition}\label{def:notation:multiview_aligned_matrices}
Given a multiview dataset $S$ with $\ell$ samples and $m$ views, 
we form a matrix for view $i$ by using horizontal concatenation:
$$ X_i := \left[ x_1^{(i)} \cdots x_\ell^{(i)} \right]. $$

We refer to the set $\{ X_1, \ldots, X_m \}$ as \emph{multiview aligned matrices}.

In general, we will say that two matrices are \emph{aligned}, if their columns form observation vector pairs, related to a multiview dataset.
\end{definition}

\begin{definition}\label{def:notation:multiview_stacked_matrix}
A \emph{multiview stacked matrix} based on a dataset $S$ with $\ell$ samples and $m$ views is a matrix obtained by
vertically concatenating the multiview aligned matrices:
$$ X := \left[ X_1; \cdots X_m; \right].$$
\end{definition}

\begin{definition}\label{def:notation:centered_matrix}
The sample matrix is centered if its rows sum to zero.
\end{definition}

\begin{definition}\label{def:notation:empirical_mean}
Given a $n\times\ell$ sample matrix $X$, the \emph{empirical mean} $\mu_X \in \RR^n$ is computed as:
$$ \mu_X(i) = \frac{1}{\ell} \sum_j X(i,j).$$
\end{definition}

\begin{definition}\label{def:notation:empirical_covariance}
Given a sample matrix $X \in n \times \ell$
 the empirical covariance is defined as:
$$ Cov(X) := \frac{1}{n-1}(X - \mu_{X} \cdot \vec{1}_\ell^T)\cdot(X - \mu_{X} \cdot \vec{1}_\ell^T)^T.$$
\end{definition}

\begin{definition}\label{def:notation:empirical_variance}
\emph{Empirical variance} $Var(X)$ is defined as the empirical covariance for single dimensional sample matrices, that is, when $n$ equals $1$.
\end{definition}

\begin{definition}\label{def:notation:empirical_cross_covariance}
Given two aligned sample matrices $X_1 \in n_1 \times \ell$ and $X_2 \in n_2 \times \ell$
 the empirical cross-covariance is defined as:
$$ Cov(X_1, X_2) := \frac{1}{n-1}(X_1 - \mu_{X_1} \cdot \vec{1}_\ell^T)\cdot(X_2 - \mu_{X_2} \cdot \vec{1}_\ell^T)^T.$$
\end{definition}


\section{Kernel Methods}
The following definitions will be relevant for our discussion of kernel versions of the methods relevant to this thesis. For definitions
of standard concepts from topology we refer the reader to standard texts~\cite{bourbaki1998general}.

\begin{definition}\label{def:notation:metric_space}
A \emph{metric space} is an ordered pair $(M,d)$, where $M$ is a set and $d: M \times M \rightarrow \RR$ is a \emph{metric} on $M$, i.e.,
a function which satisfies for all $x,y,z \in M$:
$$d(x,y) \geq 0$$
$$d(x,y) = 0 \Longleftrightarrow x = y$$
$$d(x,y) = d(y,x)$$
$$d(x,z) \leq d(x,y) + d(y,z)$$
\end{definition}

\begin{definition}\label{def:notation:cauchy_sequence}
Let $X$ be a metric space equipped with a metric $d: X \times X \rightarrow \RR$. A \emph{Cauchy sequence} is a sequence
of elements $x_1, x_2, x_3, \ldots$ if for every $\epsilon > 0$ there is a positive integer $N$ such
that for all $m,n > N$:
$$ d(x_m, x_n).$$
\end{definition}

\begin{definition}\label{def:notation:complete_space}
A metric space $X$ is complete, if every Cauchy sequence of elements in $X$ converges to an element of $X$.
\end{definition}

\begin{definition}\label{def:notation:separable_space}
A topological space $H$ is \emph{separable} if it contains a countable dense subset; that is, there
exists a sequence ${x_n}_{n=1}^{\infty}$ such that every nonempty open subset of the space contains at least
one element of the sequence.
\end{definition}

\begin{definition}\label{def:notation:hilbert_space}
A \emph{Hilbert Space} $H$ is an inner product space that is both \emph{separable} and \emph{complete}.
\end{definition}

\begin{definition}\label{def:notation:kernel_function}
Let $V \subset \RR^n$. A \emph{kernel} is a function $\kappa: V \times V \rightarrow \RR$ that satisfies:
$$ \kappa(x,y) = \lbrace \phi(x), \phi(y) \rbrace, $$
where $\phi: V \rightarrow H$ is a function from $V$ to a Hilbert space.
\end{definition}

\begin{definition}\label{def:notation:kernel_matrix}
Given a sample matrix $n\times\ell$ $X$ and a kernel function $\kappa$, we define a \emph{kernel matrix} $K \in \RR^{\ell \times \ell}$ as:
$$ K(i,j) := \kappa(x_i, x_j).$$
A special case a of kernel matrix is the $Gram$ matrix, where the standard inner product is used as the kernel function (also referred to as the \emph{linear kernel}).
\end{definition}

\begin{definition}\label{def:notation:positive_semidefinite}
A matrix $A$ is \emph{positive semidefinite} if:
$$ x^T A x \geq 0,\quad \forall x.$$
\end{definition}

