%--------------------------------------------------------------------------------------------------
%
\chapter{Notation and definitions}

We first introduce the notation we use throughout the thesis:
\begin{itemize}
\item Column vectors are denoted by lowercase letters, e.g. $x$ and matrices are denoted by uppercase letters, e.g. $X$.
\item Subscripts are used to enumerate vectors or matrices, e.g. $x_1, x_2$, $X_1$, except in the
special case of the identity matrix, $I_n$ and the zero matrix $0_{k,l}$. In these cases, the subscripts denote row and column dimensions.
\item We use superscripted symbol $T$ for vector and matrix transpose, e.g. $x^T$
\item Let $\norm{v}$ or $\norm{v}_2$ denote the $\ell_2$ norm of the vector $v$ and $\norm{A}_F$, $\norm{A}_1$ and $\norm{A}_2$ to denote the Frobenious norm and operator norms induced by $1$-norm and $2$-norm respectively.
\item MATLAB notation~\cite{golub}
\begin{itemize}
\item The $i$-th element of vector $x$ is denoted by $x(i)$ and the matrix entry in the $i$-th row and $j$-th column is denoted by $X(i,j)$.
\item The $i$-th row of matrix $X$ is denoted by $X(i,:)$ and the $j$-th column by $X(:,j)$.
 matrix elements, rows and columns {(e.g. ${X(i,j), X(i,:), X(:,j)}$)}
\item Matrix concatenation: $[A B]$ represents horizontal concatenation and $[A; B]$ represents vertical concatenation.
\end{itemize}
\item Spaces
\begin{itemize}
 \item $\RR^n$ denotes the $n$-dimensional real vector space
 \item $\RR^{n\times m}$ denotes the $(n \cdot m)$-dimensional vector space used when specifying
 matrix dimensions.
 \item $\NN$ denotes the natural numbers.
 \item $\sym_n^{+}$ denotes the space of symmetric positive definite $n$-by-$n$ matrices.
\end{itemize}
\end{itemize}

random variables, samples
n-view dataset



