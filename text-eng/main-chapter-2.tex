%--------------------------------------------------------------------------------------------------
% 
\chapter{Floating Bodies}
%--------------------------------------------------------------------------------------------------

Floating bodies are figures, tables and algorithms. 

\section{Figures}

Captions should be placed below figures as shown in \figurename~\ref{fig:ips-short}. If a caption is shorter than the line width, it should be centered. 

\begin{figure}[htb]
	\centering
		\includegraphics[width=0.4\textwidth]{figures/ipslogo-cut.pdf}
	\caption{A large IPS logo.}
	\label{fig:ips-short}
\end{figure}

On the other hand, if a caption is very long (see \figurename~\ref{fig:ips-long}), only its first (short) part should be put in the List of Figures. 

\begin{figure}[htb]
	\centering
		\includegraphics[width=0.15\textwidth]{figures/ipslogo-cut.pdf}
	\caption[A small IPS logo.]{A small IPS logo. The IPS has its own logo and a uniform graphic image, which is used on all its documents.}
	\label{fig:ips-long}
\end{figure}

\section{Tables}

Similar rules apply also to captions of tables, with the exception that captions are placed above tables (see \tablename~\ref{tab:example}).

\begin{table}[htb]
	\caption{A simple table.}
	\label{tab:example}
	\centering
		\begin{tabular}{ccc}
			\hline
			A & B & C \\
			\hline
			12 & 9834 & 327 \\
			51 & 2234 & 97 \\
			\hline
		\end{tabular}
\end{table}

\section{Algorithms}

Algorithm~\ref{alg:myalgorithm} presents an algorithm example. 

\begin{algorithm}[htb]
	\caption{An algorithm example.}
	\label{alg:myalgorithm}

	\vspace{5pt}
	\KwData{this text}
	\KwResult{complete understanding}
	\vspace{5pt}
	initialization\;
	\While{not at end of this document}{
		read current section\;
		\eIf{understood}{
			go to next section\;
			current section becomes this one\;
		}{
			go back to the beginning of current section\;
		}	
	}
\end{algorithm}